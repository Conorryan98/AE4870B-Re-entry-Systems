\section{How to properly enter code listings of any language}
Source: \url{https://www.overleaf.com/project/5ca8e08c504f2453fce240ba}

First add the commented code to the main, above begin document. Then add the listing in the folder listing, and then you can add the listing as is done in \cref{lst:value_iteration}.



\lstinputlisting[language=java,caption={myListing},label={lst:value_iteration}]{listings/listingExample.java}

\subsection{Matlab listing}
% % Create a matlab listing
% \usepackage{listings}
% \usepackage{color} %red, green, blue, yellow, cyan, magenta, black, white
% \definecolor{mygreen}{RGB}{28,172,0} % color values Red, Green, Blue
% \definecolor{mylilas}{RGB}{170,55,241}

Then inside \verb+begin documetn+:
\begin{verbatim}
\begin{document}

% Specify matlab listing style
\lstset{language=Matlab,%
    %basicstyle=\color{red},
    breaklines=true,%
    morekeywords={matlab2tikz},
    keywordstyle=\color{blue},%
    morekeywords=[2]{1}, keywordstyle=[2]{\color{black}},
    identifierstyle=\color{black},%
    stringstyle=\color{mylilas},
    commentstyle=\color{mygreen},%
    showstringspaces=false,%without this there will be a symbol in the places where there is a space
    numbers=left,%
    numberstyle={\tiny \color{black}},% size of the numbers
    numbersep=9pt, % this defines how far the numbers are from the text
    emph=[1]{for,end,break},emphstyle=[1]\color{red}, %some words to emphasise
    %emph=[2]{word1,word2}, emphstyle=[2]{style},    
}    
\end{verbatim}
